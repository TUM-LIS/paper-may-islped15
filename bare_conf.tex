\documentclass[conference]{IEEEtran}

\begin{document}

\title{Emulation-based Generation of Fine-grained Sequential Approximate Circuits }


\author{\IEEEauthorblockN{BLIND REVIEW}
\IEEEauthorblockA{}
\and
\IEEEauthorblockN{BLIND REVIEW}
\IEEEauthorblockA{}
}





% use for special paper notices
%\IEEEspecialpapernotice{(Invited Paper)}




% make the title area
\maketitle


\begin{abstract}
%\boldmath
The abstract goes here.
\end{abstract}
% IEEEtran.cls defaults to using nonbold math in the Abstract.
% This preserves the distinction between vectors and scalars. However,
% if the conference you are submitting to favors bold math in the abstract,
% then you can use LaTeX's standard command \boldmath at the very start
% of the abstract to achieve this. Many IEEE journals/conferences frown on
% math in the abstract anyway.

% no keywords




% For peer review papers, you can put extra information on the cover
% page as needed:
% \ifCLASSOPTIONpeerreview
% \begin{center} \bfseries EDICS Category: 3-BBND \end{center}
% \fi
%
% For peerreview papers, this IEEEtran command inserts a page break and
% creates the second title. It will be ignored for other modes.
\IEEEpeerreviewmaketitle



\section{Introduction}
\subsection{Approximate Computing}
\subsection{Related Work}

\section{Implementation}
\subsection{faultify - Probability-aware Fault Emulation}
\subsection{Application-reasoned Approximation }
\subsection{Approximation of Sequential Circuits}

\section{Evaluation}
\subsection{Case-study}



\section{Conclusion}


\section*{Acknowledgment}





% trigger a \newpage just before the given reference
% number - used to balance the columns on the last page
% adjust value as needed - may need to be readjusted if
% the document is modified later
%\IEEEtriggeratref{8}
% The "triggered" command can be changed if desired:
%\IEEEtriggercmd{\enlargethispage{-5in}}

% references section
\bibliographystyle{IEEEtran}
\bibliography{IEEEabrv,islped}





% that's all folks
\end{document}


